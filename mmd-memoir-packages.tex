%
%   Default packages for memoir documents created by MultiMarkdown
%
\usepackage{polyglossia} % XeTeX replacement for Babel
\setmainlanguage[variant=british]{english} % Polyglossia command

% Fonts
\usepackage{extfonts} % simple sty file which must be in your TeX path, define your fonts there

%\usepackage{fancyvrb}          % Allow \verbatim et al. in footnotes
\usepackage{graphicx}           % To include graphics in pdfs (jpg, gif, png)
\usepackage{booktabs}           % Better tables
\usepackage{tabulary}           % Support longer table cells
\usepackage[svgnames]{xcolor}   % Allow for color (annotations), see xcolor.pdf for details
\usepackage{ifthen}

% Recommended setting for Polyglossia
\usepackage[autostyle=true,english=british,autopunct=true,strict=true]{csquotes}

% Biblatex:
% ==============================================================================
% Pick one of the below to switch between MLA and Chicago styles

%\usepackage[style=mla,autocite=footnote,backref=true,backend=biber]{biblatex}

% == OR ==

% Change ‘notes’ to ‘authordate’ if you wish
\usepackage[notes,strict,autocite=footnote,backref=true,%
hyperref=true,autocite=footnote,backend=biber,bibencoding=UTF-8]%
{biblatex-chicago}

% ==============================================================================

% Support for hyperlinks, load after biblatex
\usepackage[pdfborder={0 0 0},%
colorlinks=true,linkcolor=MidnightBlue,citecolor=MidnightBlue,%
urlcolor=MidnightBlue]{hyperref}
% Set the PDF strings as Unicode
\hypersetup{unicode=true}

\def\myauthor{Author}           % In case these were not included in metadata
\def\mytitle{Title}
\def\mykeywords{}

%\def\mybibliostyle{plain}
\def\bibliocommand{}
%\VerbatimFootnotes
